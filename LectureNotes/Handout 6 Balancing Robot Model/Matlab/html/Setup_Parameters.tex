
% This LaTeX was auto-generated from MATLAB code.
% To make changes, update the MATLAB code and republish this document.

\documentclass{article}
\usepackage{graphicx}
\usepackage{color}

\sloppy
\definecolor{lightgray}{gray}{0.5}
\setlength{\parindent}{0pt}

\begin{document}

    
    \begin{verbatim}
stall_torque = 170; % oz-in
stall_current = 5; % A
Max_voltage = 12; % V
Max_RPM = 200; % rpm
Max_RPM_Current = .3; % A
Counts_per_Revolution = 3200; % Encoder counts
Value_for_full_PWM = 255; % digital output for 100% PWM
Wheel_radius = 3; % in
l = .38; % m
m1 = .6; % kg
m2 = 1.9; % kg
b_phi=0; % rotational friction
b_x=0; % translational friction
La=0; % Motor inductance
Ts = .001; % Sample time of controller
phi_0 = pi % initial condition for angle of pendulum

%
% Conversions
%
r=Wheel_radius/39.37; % convert in to m
tau_max=stall_torque/141.612; % convert oz-in to Nm
%
% Motor Constants
%
Kt = tau_max/stall_current;    % divide by stall current to get motor constant
Ra = Max_voltage/stall_current; % motor resistance
Ke = (Max_voltage - Ra*Max_RPM_Current)/(Max_RPM*2*pi/60); % back emf constant
%
Kv = Max_voltage/Value_for_full_PWM; % digial output to V conversion for motor
Ktheta = 2*pi/Counts_per_Revolution;
\end{verbatim}

        \color{lightgray} \begin{verbatim}
phi_0 =

    3.1416

\end{verbatim} \color{black}
    


\end{document}
    
